\documentclass[aps,pra,notitlepage,amsmath,amssymb,letterpaper,12pt]{revtex4-1}
\usepackage{amsthm}
\usepackage{graphicx}
%  Above uses the Americal Physical Society template for Physical Review A
%  as a reasonable and fully-featured default template
 
%  Below define helpful commands to set up problem environments easily
\newenvironment{problem}[2][Problem]{\begin{trivlist}
\item[\hskip \labelsep {\bfseries #1}\hskip \labelsep {\bfseries #2.}]}{\end{trivlist}}
\newenvironment{solution}{\begin{proof}[Solution]}{\end{proof}}
 
% --------------------------------------------------------------
%                   Document Begins Here
% --------------------------------------------------------------
 
\begin{document}
 
\title{CS510 CW 1}
\author{Evan Walker and Ehsan Yaghmaei}
\affiliation{CS 510, Schmid College of Science and Technology, Chapman University}
\date{\today}

\maketitle

\section{Derivative Definition} % Specify main sections this way

% 1 is the problem number
\begin{problem}{} 
How to define a derivative of a function.
\end{problem}
 
\textbf{Definition}  %You can also use proof in place of solution
The derivative of  $f(x)$ with respect to x is the function $f'(x)$ and is defined as,
\begin{align}
$f'(x)$ &= $\lim_{h \to 0}\frac{f(x + h) - f(x)}{h} \\
&= -\frac{\hbar^2}{2m}\nabla^2\psi(x,t) + V(x)\psi(x,t). \nonumber
\end{align}
% Use align environments for equations. The \\ is a newline character. The & is the alignment character.
% Using align* or \nonumber on each line removes equation numbers
\end{solution}

\subsection{Find a Derivative} % Specify subsections and subsubsections this way

To find the derivative of a function y = f(x) we use the slope formula:

\begin{align}
$f'(x)$ &= $\lim_{h \to 0}\frac{f(x + h) - f(x)}{h} \\
&= -\frac{\hbar^2}{2m}\nabla^2\psi(x,t) + V(x)\psi(x,t). \nonumber
\end{align}

x changes from x to x+$\nabla$x
y changes from $f(x)$ to $f(x + $\nabla$x)

Then
Fill in this slope formula:
\begin{align}
$frac{$\nabla$x}{$\nabla$y} &= $frac{f(x +$\nabla$x)) - f(x)}{h} \\ \\
    
    Simplify it as best we can
    
    Then make shrink towards zero
\begin{figure}[h!] % h forces the figure to be placed here, in the text
  \includegraphics[width=0.4\textwidth]
  {GodfreyKneller-IsaacNewton-1689.jpg}  % if pdflatex is used, jpg, pdf, and png are permitted
  \caption{Around the 1670s, Sir Isaac Newton's conceptual understanding of physics prompted him to invent the complicated math known as calculus.}
  \label{fig:figlabel}
\end{figure}

This text should be below the figure unless \LaTeX  decides that a different layout works better.
 
% Repeat as needed
 
\end{document}
